\documentclass{beamer} % gli diamo la classe delle presentazioni 
\usepackage{graphicx} % serve per inserire immagini
\usepackage{multicol} % useremo dopo per fare le colonne
\usepackage{amssymb} 


% Vogliamo cambiare lo stile della presentazione, ci sono matrici con stili diversi che combinano città (=stile) e con colori
\usetheme{Szeged}
\usecolortheme{beaver} % posso anche usare beaver o dove come colori, ce ne sono tanti 
% Tutto rimane così com'è ma cambia colore o struttura, non si scombina nulla

% Prepariamo quello che diventerà titolo 
\title{Analisi vegetazione delta del Volga (RU)}
\author{Luca Fontanabona}
\date{Giugno 2024}

% Le slides sono frame, dobbiamo inserirle 

\begin{document}

\maketitle % è un modo prefissato per fare il titolo, serve per avere titoli uniformi 

\AtBeginSection[] % fa tutto quello che gli dico ad ogni nuova sessione
{
\begin{frame}
\frametitle{Outline}
\tableofcontents[currentsection] % gli dico di fare un indice che metterà in una slide all'inizio di ogni sezione evidenziando la sezione nuova che inizia 
\end{frame}
}


\section{Introduzione}


\begin{frame}{Inquadramento dell'area} % titolo della slide tra aprentesi {} all'inizio
\begin{multicols}{2} % funzione che crea le colonne, devo mettergli tra parentesi {} quante 
Dove ci troviamo? \\
Russia, Oblast di Astrachan'\\
\bigskip
Quale periodo analizziamo? \\
	\begin{itemize}
 		\item 2019, 2021, 2023
   		\item Immagini di: 
     			\begin{enumerate}
				\item fine marzo
				\item seconda metà di maggio
				\item seconda metà di giungo
				\item fino agosto
			\end{enumerate}
	\end{itemize}	 
\columnbreak % serve per dirgli che voglio iniziare la nuova colonna
	\begin{figure}[h]
        	\centering
        	\includegraphics[width=0.9\linewidth]{Astrakhan_Oblast.png}
        	\label{fig:1}
    	\end{figure}
\end{multicols}
\end{frame}


\begin{frame}{Considerazioni ambientali} 
Clima desertico: 
	\begin{itemize}
        	\item Bassissima piovosità
        	\item Grande escursione termica annuale 
	\end{itemize}
Vegetazione assente in quasi tutto il periodo dell'anno 

\begin{multicols}{2}
	\begin{figure}
		\centering
		\includegraphics[width=0.9\linewidth]{ClimogrammaAstrachan'.jpg}
		\label{fig:enter-label}
	\end{figure}
\columnbreak
	\begin{figure}
		\centering
		\includegraphics[angle=-90, width=0.6\linewidth]{1718112587500.jpg}
		\label{fig:enter-label}
	\end{figure}
\end{multicols}
\end{frame}


\section{Obiettivo}


\begin{frame}{Ipotesi iniziale e possibili impatti nel tempo}
Presenza di vegetazione maggiore in primavera (maggio giugno)
Quasi totale assenza di vegetazione negli altri periodi. \\
\bigskip 
Possibile impatto dell'uomo nel tempo, da indagare nel momento di massima vegetazione
\end{frame}


\section{Materiali e metodi}


\begin{frame}{Immagini satellitari}
\begin{multicols}{2}
Immagini da Copernicus\\
L'area comprende: 
	\begin{itemize}
		\item Due rami del Volga 
		\item Alcune aree pascolate e coltivate 
		\item Molte aree naturali
	\end{itemize}
\columnbreak
	\begin{figure}
		\centering
		\includegraphics[width=0.8\linewidth]{inquadramento .png}
  		\label{fig:enter-label}
	\end{figure}
\end{multicols}
\end{frame}

% Creo un nuovo comando per allargare il margine, mi serve dopo per farci stare le tre colonne. 
\newcommand{\smallermargins}{
  \newgeometry{
    left=0.25in,
    right=0.25in,
    top=1in,
    bottom=1in,
    includehead,
    includefoot
  }
}

\begin{frame}
\smallermargins % uso il comando appena creato per ridurre i margini 
\frametitle{Analisi statistiche}
\begin{multicols}{3} % questa volta sono 3 colonne
Pacchetti usati: 
	\begin{itemize} 
		\item imageRy
  		\item terra
		\item ggplot2
		\item viridis
        	\item fields
	\end{itemize}
\columnbreak
\pause Funzioni principali: % uso \pause per fare in modo che compaia dopo, oer rendere le slide più dinamiche
	\begin{itemize}
        \setbeamertemplate{itemize item}{\textbullet} % cambia il simbolo di itemize, per cambiare da quello predefinito
		\item colorRampPalette
		\item rast
		\item im.plotRGB
		\item plot 
		\item im.ndvi
		\item im.pca
	\end{itemize}
 \columnbreak  
	\begin{itemize}
        \setbeamertemplate{itemize item}{\textbullet}
        \bigskip
		\item focal
		\item image.plot
		\item im.classify
		\item matrix
		\item write.csv2
		\item ggplot
	\end{itemize}
\end{multicols}
\end{frame}


\restoregeometry % serve per far tornare i margini com'erano prima, altrimenti sembra che restino cambiati anche nelle slide successive


\begin{frame}{Immagini di partenza}
False colors (3 bande: nir, rosso, verde) 
	\begin{figure}
		\centering
		\includegraphics[width=0.75\linewidth]{False colors.png}
		\label{fig:enter-label}
	\end{figure}
\end{frame}


\begin{frame}{Immagini di partenza}
Banda del NIR con colori adatti ai daltonici
	\begin{figure}
		\centering
		\includegraphics[width=0.75\linewidth]{NIR.png}
		\label{fig:enter-label}
	\end{figure}
\end{frame}


\section{Risultati}


\begin{frame}{Risultati NDVI}
	\begin{equation} % serve per inserire le equazioni, nel nostro caso è una formulina estremamente banale
		{NDVI}= \frac{nir-red}{nir+red}
	\end{equation} 
	\begin{figure}
		\centering
		\includegraphics[width=0.65\linewidth]{NDVI.png}
		\label{fig:enter-label}
	\end{figure}
\end{frame}


\begin{frame}{Risultati}
Escludiamo marzo \\
\bigskip
Sono state calcolate le differenze tra immagini per avere una visione migliore su come varia il valore di NDVI 
	\begin{equation}
		Giugno - Maggio
	\end{equation}
	\begin{equation}
		Giugno - Agosto
	\end{equation}
\end{frame}


\begin{frame}{Risultati differenze}
	\begin{figure}
		\centering
		\includegraphics[width=0.7\linewidth]{differenze gm ga.png}
		\label{fig:enter-label}
	\end{figure}
\centering % si usa per centrare il testo
Problema: scala non è la stessa!
\end{frame}


\begin{frame}{Risultati differenze}
Scala uguale, vediamo che giugno è mese con maggiore vegetazione (valori di sottrazione positivi)
	\begin{figure}
		\centering
		\includegraphics[width=0.8\linewidth]{differenze con stessa scala.png}
		\label{fig:enter-label}
	\end{figure}
\end{frame}


\begin{frame}{Diversità}
Operazioni svolte: 
	\begin{itemize}
		\item PCA
		\item standard deviation su componente principale 
		\item rappresentazione grafica del risultato
	\end{itemize}
\end{frame}


\begin{frame}{Risultato PCA + sd}
	\begin{figure}
		\centering
		\includegraphics[width=0.7\linewidth]{standard deviation.png}
		\label{fig:enter-label}
	\end{figure}
\centering 
Di nuovo problema della scala 
\end{frame}


\begin{frame}{Risultato PCA + sd}
Scala sistemata
	\begin{figure}
		\centering
		\includegraphics[width=0.8\linewidth]{standard deviation scala giusta.png}
  		\label{fig:enter-label}
	\end{figure}
\end{frame}


\begin{frame}{Cambiamenti nella vegetazione}
Obiettivo: indagare se nell'arco di tempo studiato ci sia stato un cambiamento nella vegetazione \\
Operazione: 
	\begin{equation}
		Giugno 2019 - Giugno 2023
  	\end{equation}
\end{frame}


\begin{frame}{Risultato}
	\begin{figure}
		\centering
		\includegraphics[width=0.85\linewidth]{2019-2023.png}
  		\label{fig:enter-label}
	\end{figure} 
\end{frame}


\begin{frame}{Classificazione}
\begin{multicols}{2}
Per mostrare meglio: 
	\begin{itemize}
		\item Classificazione dell'NDVI
		\item Calcolo delle frequenze
		\item Tabella e grafico 
  	\end{itemize}
\columnbreak
	\begin{figure}
		\centering
		\includegraphics[width=0.95\linewidth]{classificazione NDVI.png}
  		\label{fig:enter-label}
	\end{figure}
\end{multicols}
\end{frame}


\begin{frame}{Risultati classificazione}
	\begin{figure}
    		\centering
    		\includegraphics[width=0.4\linewidth]{tabella percentuali immagine.png}
    		\includegraphics[width=0.7\linewidth]{plot percentuali.png}        
    		\label{fig:enter-label}
	\end{figure}
\end{frame}


\section{Conclusioni}


\begin{frame}{Conclusioni}
	\begin{enumerate}
        	\item Momento con maggiore diversità e vegetazione è giugno 
        	\item \pause Non si rileva grande differenza tra 2019 e 2023, forse dovuto a: \\
       			\begin{itemize}
            			\item \pause grande influenza della stagionalità nella vegetazione 
            			\item \pause arco temporale troppo corto per notare cambiamenti climatici
        		\end{itemize}
   	\end{enumerate}
\end{frame}


% Andiamo a impostare un'immagine come sfondo per l'ultima slide per renderla più carina
\setbeamertemplate{background}{
        \includegraphics[width=\paperwidth, height=\paperheight, keepaspectratio]{photo1718705873.jpeg}
        }
        
\begin{frame}{}
\bigskip % non sarà il modo più elegante ma funziona
\bigskip
\bigskip
\bigskip
\bigskip
\bigskip
\bigskip
\bigskip
\bigskip
\bigskip
\bigskip
\bigskip
\bigskip
\bigskip
\bigskip
\centering {\textcolor{white}{\Huge Grazie per l'attenzione!}} \\
\smallskip
\textcolor{white}{GitHub: \url{https://github.com/LucaFontanabona}}
\end{frame}


\end{document}
