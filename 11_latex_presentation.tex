\documentclass{beamer} % gli diamo la classe delle presentazioni 
\usepackage{graphicx} % serve per inserire immagini

% Vogliamo cambiare lo stile della presentazione, ci sono matrici con stili diversi che combinano città (=stile) e con colori
\usetheme{}
\usecolortheme{crane} % posso anche usare beaver o dove come colori, ce ne sono tanti 
% Tutto rimane così com'è ma cambia colore o struttura, non si scombina nulla

\title{Presentazione in LaTex}
\author{Luca Fontanabona}
\date{maggio 2024}

% Le slides sono frame, dobbiamo inserirle 

\begin{document}

\maketitle % è un modo prefissato per fare il titolo, serve per avere titoli uniformi 

\AtBeginSection[] % fa tutto quello che gli dico ad ogni nuova sessione
{
\begin{frame}
\frametitle{Outline}
\tableofcontents[currentsection] % gli dico di fare un indice che metterà una slide all'inizio di ogni sezione evidenziando la sezione nuova che sto iniziando
\end{frame}
}

\section{Introduzione}

\begin{frame}{My first slide} % Il titolo va tra graffe
qua posso inserire il testo % testo inserito nella slide
\end{frame}

\begin{frame}{Itemize}
Quello di cui andremo a parlare in questa persentazione è: 
    \begin{itemize}
        \item Primo argomento
        \item secondo argomento 
        \item terzo argomento 
\end{itemize}
\end{frame}

%voglio metterli come successivi, non tutti assieme appena arrivo alla slide. Crea molte slide diverse che poi fanno da animazione
\begin{frame}{Itemize}
Quello di cui andremo a parlare in questa persentazione è: 
    \begin{itemize}
        \item Primo argomenot 
        \pause \item  secondo argomento 
        \pause \item  terzo argomento 
\end{itemize}
\end{frame}


% Dimensioni del testo, può essere utile cambiarla. Usiamo delle funzioni per farlo. Ogni grandezza (in pt) ha un codice, una funzione che mi da la grandezza

\begin{frame}{Prove di grandezza}
Dovrebbe venire  \Huge{Enorme}
\end{frame}

\begin{frame}{Andare a capo}
Vado a capo con doppio backslash \\
Ora di nuovo \\
\bigskip 
Ora dovrei avere uno spazio maggiore 
\end{frame}

\begin{frame}{Grassetto ma non solo}
Alcune volte potrebbe essere utile avere qualcosa scritto in \textbf{grassetto} per evitare che le persone si sconcentrino leggendo. Stessa cosa posso fare con \textit{corsivo} per mostrare meglio
\end{frame}

% vogliamo inserire le formule, c'è wikibook con tutte le info su come trovare il codice 
\section{Formulas}
\begin{frame}{Algoritmi} 
Formula della deviazione standard \\
\bigskip
\centering % per fare in modo che venga al centro 
    \delta = \sqrt{\frac{\displaystyle\sum_{i=1}^{N}{(x- \mu)^2}}{N}} 
% Scriviamo la funzione, partiamo dalla frazione, inseriamo quello che ci va (numeratore e denominatore), mettiamo poi la sommatoria e l'elevamento alla 2. inseriamo anche il pedice e l'apice della sommatoria e vogliamo che siano sopra, ci mettiamo quindi la funzione \displaystyle per farlo. 
% Radice quadrata al di fuori di tutto. Per avere simboli strani invece che sd o mu devo mettere lo slash prima che glielo fa riconoscere come tale. 
\end{frame}

\section{Results} % Per ora non le vedo (??)
\begin{frame}{Risultati principali} % vogliamo avere un'immagine 
    \begin{figure}
        \centering
        \includegraphics[width=0.9\linewidth]{space.jpg}
        % \caption{Enter Caption} perchè di solito non si usa
        \label{fig:enter-label}
\end{figure}
\end{frame}

\begin{frame}{Risultati principali ma due immagini} % vogliamo mettere due immagini questa volta 
    \begin{figure}
        \centering
        \includegraphics[width=0.4\linewidth]{space.jpg}
        \includegraphics[width=0.4\linewidth]{space.jpg}
        % \caption{Enter Caption} perchè di solito non si usa
        % \label{fig:enter-label} in questo caso non ci serve
\end{figure}
\end{frame}

% Ora voglio metterne 4 invece che due 

\begin{frame}{Risultati principali ma 4immagini} % Vogliamo mettere 4 immagini questa volta, due sopra e due sotto
    \begin{figure}
        \centering
        \includegraphics[width=0.4\linewidth]{space.jpg}
        \includegraphics[width=0.4\linewidth]{space.jpg} \\ % il trucco sta nel mettere l'accapo al punto in cui vogliamo andare a capo
        \includegraphics[width=0.4\linewidth]{space.jpg}
        \includegraphics[width=0.4\linewidth]{space.jpg} \\ % vado a capo perchè voglio metterci delle cose sotto ancora 
        % \caption{Enter Caption} perchè di solito non si usa
        % \label{fig:enter-label} in questo caso non ci serve
\end{figure}
\bigskip % se per caso voglio metterci uno spazio più grande. Anche se c'è \\ governa bigskip
\centering
\scriptsize{Rocchini et al. 2024} % metto scriptsize per rimpicciolire la scritta, importante tenere le dimensioni uguali e coerenti per tutta la presentazione
\end{frame}

% Tentativo di fare delle colonne anche se non ha funzionato molto, dobbiamo guardare su Git il codice corretto

\begin{frame}{Colonne}
    \begin{columns}
    \begin{column}{width=0.5\textwidth}
        un po' di testo o anche molto come si vuole 
    \end{column}
    \begin{column}{width=0.5\textwidth}
          \includegraphics[width=0.5\linewidth]{space.jpg}
    \end{column}
    \end{columns}
\end{frame}

% codice recuperato da GitHub del prof alla data 15/05 ancora non funziona
\begin{frame}{Columns}
    \begin{columns}
        \begin{column}{width=.5\textwidth}
            A small amount of text  here or a larger one as we wish...
        \end{column}
        \begin{column}{width=.5\textwidth}
            \includegraphics[width=.4\linewidth]{stats.png}
        \end{column}
    \end{columns}
\end{frame}


\end{document}




\end{document}
